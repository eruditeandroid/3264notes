\documentclass{article}

\usepackage{notes}

\title{3264 and All That Reading Course Notes}

\begin{document}

\maketitle

\tableofcontents

\section{9/18 (Lizzie Pratt) -- Intro to Chow Groups}

All schemes mentioned will be implicitly separated and of finite type over $\CC$.
The term ``variety'' will refer to an integral scheme (satisfying the above conditions).

\subsection{Chow groups}

\begin{dfn}
	For a scheme $X$, define groups:
	\begin{itemize}
		\item The cycle group $Z(X)$ is the free abelian group generated by classes $[Y]$ for $Y \subset X$ a closed subvariety.
		\item The rational equivalence group $\Rat(X)$ is the subgroup of $Z(X)$ generated by $[\Phi \cap (\{ t_0 \} \times X)] - [\Phi \cap \{ t_1 \times X \}]$ for $\Phi \subset \PP^1 \times X$ a closed subvariety not contained in any one fiber $\{ t \} \times X$.
		\item The Chow group is $A(X) = Z(X) / \Rat(X)$.
	\end{itemize}
\end{dfn}

\begin{prop}
	The Chow group is graded by dimension, i.e.\ $A(X) = \oplus_k A_k(X)$ where $A_k(X)$ consists of rational equivalence classes of $k$-dimensional cycles.
\end{prop}

This relies on the following theorem of commutative algebra:

\begin{thm}[Krull principal ideal theorem]
	Let $R$ be a noetherian ring and $I$ an ideal generated by $n$ elements.
	Then the codimension of $\Spec R/I \subset \Spec R$ is less than or equal to $n$.
\end{thm}

\begin{proof}[Proof of Proposition]
	It suffices to show that $\Rat(X)$ is graded by dimension.
	Let $\Phi$ be a $k$-dimensional closed subvariety of $\PP^1 \times X$ which is not contained in any fiber.
	By the principal ideal theorem, $\Phi \cap (\{ t_i \} \times X)$ has codimension $\leq 1$ in $\Phi$.
	Because $\Phi$ is not contained in any fiber, the codimensions here are nonzero.
	Thus $\Phi \cap (\{ t_i \} \times X)$ has dimension $k-1$ for each $i$.
\end{proof}

We can also grade Chow groups by codimension: for $X$ equidimensional, let $A^k(X) = A_{\dim X - k}(X)$.

\subsection{Intersection product}

We would like to define an intersection product on Chow groups sending $A^k \times A^\ell \to A^{k + \ell}$.
In the case of ``generically transverse'' intersections, this will satisfy
\[
	[A] \cdot [B] = [A \cap B].
\]
This property may fail for more general intersections.
Let us make sense of the term ``generically transverse:''

\begin{dfn}
	Let $A$ and $B$ be subvarieties of a smooth variety $X$.
	We say that $A$ and $B$ intersect transversely at $p \in A \cap B$ if
	\[
		\codim T_p(A \cap B) = \codim T_p A + \codim T_p B.
	\]
	We also say $A$ and $B$ are generically transverse if they are transverse on a dense open of $A \cap B$.
\end{dfn}

\begin{thm}
	Let $X$ be a smooth quasi-projective variety.
	Then there exists a product structure making $A(X) = \oplus_c A^c(X)$ into a graded commutative ring.
\end{thm}

This is a consequence of:

\begin{lem}[Moving lemma]
	Let $X$ be as above.
	\begin{itemize}
		\item For $\alpha, \beta \in A(X)$, there exist generically transverse cycles $A, B \in Z(X)$ with $\alpha = [A]$ and $\beta = [B]$.
		\item The class $[A \cap B]$ is independent of the choice of $A$ and $B$.
	\end{itemize}
\end{lem}

One can show that smoothness is necessary in order for this to work.

\subsection{The class group}

\begin{prop}
	Let $X$ be an $n$-dimensional variety.
	Then $A_{n-1}(X) = \Cl(X)$, the Weil divisor class group.
\end{prop}

Recall that we have a comparison map
\[
	\divop: \Gamma(X, \Kc^\times / \Oc^\times) \to Z_{n-1}(X)
\]
which induces a commutative diagram
\[
	\begin{tikzcd}
		\CaCl(X) \rar["\divop"] \dar["\Oc_X(-)", swap] & \Cl(X). \\
		\Pic X \ar[ur, "c_1", swap] &
	\end{tikzcd}
\]
In favorable cases, these maps are isomorphisms:
\begin{itemize}
	\item $\Oc_X(-)$ is always injective and is surjective if $X$ is projective over $k$ or integral.
	\item $\divop$ and $c_1$ are not necessarily injective or surjective in general.
\end{itemize}

\section{9/25 (Lizzie Pratt) -- Chow Rings}

\subsection{Chow ring of $\AA^n$}

\begin{prop}
	The Chow ring of $\AA^n$ is $\ZZ \cdot [\AA^n]$.
\end{prop}

\begin{proof}
	We need to show that, for every $Y \subsetneq \AA^n$, there exists a rational equivalence $\langle Y \rangle \tilde 0$.
	WLOG we may assume $0 \not\in Y$, so that there exists $g \in I(Y)$ such that $g$ has nonzero constant term.
	Let
	\[
		W = \{ (t, ty) \, | \, t \in \GG_m, y \in Y \} \subset \GG_m \times \AA^n.
	\]
	Then $W^0_t = t Y$ for $t \in \GG_m$.
	Let $W = \ol{W^0} \subset (\PP^1 \setminus 0) \times \AA^n$.
	Then $G(t, z) := g(z/t)$ gives a function on $(\PP^1 \setminus 0) \times \AA^n$ which vanishes on $W^0$, hence also on $W$.
	As $t \to \infty$, we have $g(z/t) \to g(0) \neq 0$, hence $W_\infty = \emptyset$ and $\langle Y \rangle \sim 0$.
\end{proof}

\subsection{Mayer-Vietoris and Excision}

\begin{prop}
	\begin{itemize}
		\item (Mayer-Vietoris) Let $X_1$ and $X_2$ be closed subschemes of $X$.
			For all $k$, there exists an exact sequence
			\[
				\begin{tikzcd}
					A_k(X_1 \cap X_2) \rar & A_k(X_1) \oplus A_k(X_2) \rar & A_k(X_1 \cup X_2) \rar & 0.
				\end{tikzcd}
			\]
		\item (Excision) Let $Y \subset X$ be a closed subscheme.
			There exists a (graded) exact sequence
			\[
				\begin{tikzcd}
					A_k(Y) \rar & A_k(X) \rar & A_k(X \setminus Y) \rar & 0.
				\end{tikzcd}
			\]
	\end{itemize}
\end{prop}

\begin{proof}
	We can define $A_k(X)$ via the exact sequence
	\[
		\begin{tikzcd}
			Z_k(\PP^1 \times X) \rar["\partial"] & Z_k(X) \rar & A_k(X) \rar & 0
		\end{tikzcd}
	\]
	where
	\[
		\partial(\Phi) = \begin{cases}
			0 & \Phi \textrm{ contained in a single fiber} \\
			\langle \Phi \cap (\{ t_0 \} \times X \rangle - \langle \Phi \cap (\{ t_1 \} \times X \rangle & \textrm{otherwise.}
		\end{cases}
	\]
	Here $\im \partial = \Rat(X)$.
	For each of Mayer-Vietoris and excision, we can set up a $3 \times 3$ commutative square and apply the nine lemma to get the desired sequences.
	The diagram for excision looks like
	\[
		\begin{tikzcd}
			0 \rar & Z(Y \times \PP^1) \rar \dar & Z(X \times \PP^1) \rar \dar & Z(U \times \PP^1) \rar \dar & 0 \\
			0 \rar & Z(Y) \rar \dar & Z(X) \rar \dar & Z(X \setminus Y) \rar & 0 \\
			& A(Y) \rar \dar & A(X) \rar \dar & A(X \setminus Y) \rar \dar & 0 \\
			& 0 & 0 & 0. &
		\end{tikzcd}
	\]
\end{proof}

To continue the exact sequences further to the left, we would have to consider ``higher Chow groups.''

\subsection{Functoriality}

Here is a motivating example / exercise:
Suppose we have a morphism $f: X' \to X$ which is finite and flat (of some degree $d$).
In the flat case, we have pullback $f^*$.
In the finite (more generally, proper) case, we have pushforward $f_*$.
The composite $f_* f^*: A(X) \to A(X)$ is multiplication by $d$.

To define pushforward along a proper map, we would like to say that $f_*: [A] \mapsto [f(A)]$.
However, this is not quite correct: we need to account for multiplicity.
For the general case, we note that if $\dim A = \dim f(A)$, then $K(A)$ is a finite extension of $K(f(A))$.

\begin{dfn}
	Let $f: X \to Y$ be a proper map.
	We define $f_*: A(X) \to A(Y)$ as follows.
	\begin{itemize}
		\item If $\dim f(A) < \dim A$, then $f_*[A] = 0$.
		\item Otherwise, $f_*[A] = [K(A) : K(f(A))] [f(A)]$.
	\end{itemize}
\end{dfn}

\begin{ex}
	Let $C$ be a projective curve over $\CC$.
	Then $C \to \Spec \CC$ gives a map $A_0(C) \to \ZZ$.
	This agrees with the degree map for smooth curves (but may fail to do so in the presence of singularities).
\end{ex}

We can also define pullback along a flat map.

\begin{dfn}
	Let $f: X \to Y$ be flat.
	We define $f^*: A(X) \to A(Y)$ via $f^*[Z] = [f\inv(Z)]$ (at least in nice cases?).
\end{dfn}

\section{10/2 (Rose Lopez) -- The First Chern Class of Line Bundles}

Let $\Lc$ be a line bundle on a variety $X$, and let $\sigma$ be a nonzero rational section of $\Lc$.
On an affine open $U$, we can write $\sigma|_U$ as a rational function $f / g$.
We define the divisor of $\sigma|_U$ to be $\divop(\sigma|_U) = \divop(f) - \divop(g)$ and glue this to a global divisor $\divop(\sigma)$ on $X$.
If we choose a different rational section $\tau$, then $\alpha = \sigma / \tau$ is a rational function, and we have 
\[
	\divop(\sigma) - \divop(\tau) = \divop(\alpha) \in \Rat(X).
\]
In fact, we have the following.

\begin{prop}
	If $X$ is any scheme, then the group $\Rat(X) \subset Z(X)$ is generated by all divisors of rational functions on all subvarieties of $X$.
\end{prop}

\begin{dfn}
	Let $\Lc$ be a line bundle on a quasi-projective scheme $X$.
	The \emph{first Chern class} of $\Lc$ is
	\[
		c_1(\Lc) = [\divop(\sigma)] \in A(X)
	\]
	for any rational section $\sigma$ of $\Lc$.
\end{dfn}

\begin{ex}
	Let $X = \PP^n$, $\Lc = \Oc_{\PP^n}(d)$ with $d \geq 0$.
	We can take our rational section $\sigma$ to be any nonzero degree $d$ polynomial on a standard open affine $\AA^n$.
	It follows that $c_1(\Lc)$ is represented by the class of any degree $d$ hypersurface in $\PP^n$.
\end{ex}

\begin{prop}
	Let $X$ be a variety of dimension $n$.
	Then $c_1$ gives a group homomorphism $\Pic X \to A_{n-1}(X)$.
	If $X$ is smooth, this is an isomorphism with inverse given by $[Y] \mapsto \Oc_X(Y) = \Ic_Y\inv$ (the inverse of the ideal sheaf of $Y$).
\end{prop}

\begin{proof}
	If $\Lc, \Lc'$ are line bundles with rational sections $\sigma, \sigma'$ (respectively), then $\sigma \otimes \sigma'$ is a rational section of $\Lc \otimes \Lc'$, and $\divop(\sigma \otimes \sigma') = \divop(\sigma) + \divop(\sigma')$.
	In the smooth case, all local rings are UFDs, so all codimension one subvarieties are Cartier divisors, and these uniquely determine line bundles via the above formula.
	The preceding proposition shows that this is well-defined under the relevant notions of equivalence.
\end{proof}

In general, $c_1$ is neither injective nor surjective.
We illustrate this with some examples.

\begin{ex}
	Let $X$ be the nodal cubic.
	Then $c_1: \Pic X \to A_0(X)$ is not injective.
	If $\pi: \tilde{X} \to X$ is the normalization (so $\tilde{X} = \AA^1$), we obtain an exact sequence
	\[
		\begin{tikzcd}
			0 \rar & \oplus_{p \in X} \tilde{\Oc_p} / \Oc_p \rar & \Pic X \rar & \Pic \tilde{X} \rar & 0,
		\end{tikzcd}
	\]
	and we can use this to disprove injectivity.
	Specifically, $\Pic X$ should be $\GG_m$ and $A_0(X)$ should be $0$.
\end{ex}

\begin{ex}
	We can also find an example where $c_1$ is not surjective by looking at the rational cone.
	The salient point is that the map from Cartier divisors to Weil divisors is not surjective.
\end{ex}

\section{10/9 (Rose Lopez) -- More on $c_1$}

Recall that, if $\Lc$ is a line bundle on a quasiprojective scheme $X$, we define
\[
	c_1(\Lc) = [\divop(\sigma)] \in A^1(X)
\]
where $\sigma$ is any rational section of $\Lc$.

\begin{prop}
	Let $X$ be a smooth quasiprojective variety, $\Lc$ a line bundle on $X$, and $Y_1, \dots, Y_n$ subvarieties of $X$.
	Then there exists a cycle in $c_1(\Lc)$ that is generically transverse to each $Y_i$.
	Furthermore,
	\[
		c_1(\Lc) \cdot [Y_i] = c_1(\Lc|_{Y_i})
	\]
	for all $i$.
\end{prop}

The proof is an application of Bertini's theorem.

\subsection{The canonical class}

Let $X$ be a smooth $n$-dimensional variety.
The canonical bundle $\omega_X = \bigwedge^n \Omega_X$ has sections given by regular $n$-forms on $X$.
The \emph{canonical class} $K_X$ is defined as
\[
	K_X = c_1(\omega_X) \in A^1(X).
\]

\begin{ex}
	Let $X = \PP^n = \Proj k[X_0, \dots, X_n]$.
	On $U_0 = \{ X_0 \neq 0 \} = \{ [1 : x_1 : \dots : x_n] \}$, we consider the local section $\sigma = dx_1 \wedge \dots \wedge dx_n$ of $\omega_X$.
	Note that $\sigma$ has no zeroes or poles on $U_0$.
	Changing to the chart $U_n = \{ X_n \neq 0 \}$ with standard coordinates $y_i$, we can write
	\[
		x_i = \begin{cases}
			y_i / y_0 & i = 1, \dots, n-1 \\
			1 / y_0 & i = n
		\end{cases}
	\]
	and therefore
	\[
		dx_i = \begin{cases}
			\frac{y_0 dy_i - y_i dy_0}{y_0^2} & i = 1, \dots, n-1 \\
			-\frac{1}{y_0^2} dy_0 & i = n.
		\end{cases}
	\]
	Thus
	\[
		\sigma|_{U_n} = \frac{(-1)^n}{y_0^{n+1}} dy_0 \wedge \dots \wedge dy_{n-1}.
	\]
	It follows that $K_{\PP^n} = (-n-1) [H]$, where $[H]$ is the class of a hyperplane in $\PP^n$.
\end{ex}

\subsection{The adjunction formula}

Let $X$ be a smooth $n$-dimensional variety and $Y$ a smooth $(n-1)$-dimensional subvariety.
There is a normal bundle sequence
\[
	\begin{tikzcd}
		0 \rar & \Tc_Y \rar & \Tc_X|_Y \rar & \Nc_{Y/X} \rar & 0.
	\end{tikzcd}
\]
Taking top exterior powers, we obtain $\bigwedge^n \Tc_X|_Y \cong \bigwedge^{n-1} \Tc_Y \otimes \Nc_{Y/X}$.
Dualizing this gives $\omega_X|_Y \cong \omega_Y \otimes \Nc_{Y/X}^\vee$.

We can also consider the exact sequence
\[
	\begin{tikzcd}
		0 \rar & \Ic_{Y/X} / \Ic_{Y/X}^2 \rar["\delta"] & \Omega_{X/Y} & \Omega_Y \rar & 0,
	\end{tikzcd}
\]
which yields $\Ic_{Y/X} / \Ic_{Y/X}^2 \otimes \omega_Y \cong \omega_X|_Y$.
Thus $\Nc_{Y/X}^\vee \cong \Ic_{Y/X} / \Ic_{Y/X}^2$.
If $Y$ is a Cartier divisor, we have $\Ic_{Y/X} = \Oc_X(-Y)$, and therefore $\Ic_{Y/X} / \Ic_{Y/X}^2 = \Oc_Y(-Y)$ (we can intuitively understand this as coming from intersecting $Y$ with itself).
Combining this with the preceding yields

\begin{prop}[Adjunction formula]
	If $Y \subset X$ is a smooth $(n-1)$-dimensional subvariety of a smooth $n$-dimensional variety, then
	\[
		\omega_Y \cong \omega_X|_Y \otimes \Oc_X(Y)|_Y = \omega_X(Y)|_Y.
	\]
	In particular, if $Y$ is a smooth curve in a smooth complete surface, then
	\[
		\deg K_Y = \deg ((K_X + [Y]) \cdot [Y]).
	\]
\end{prop}

\begin{ex}
	Let $X \subset \PP^n$ be a smooth hypersurface of degree $d$.
	Then 
	\[
		\omega_X \cong \Oc_X(-n-1) \otimes \Oc_X(d) = \Oc_X(d-n-1).
	\]
	More generally, if $X$ is a smooth complete intersection of hypersurfaces $X_1, \dots, X_r$ in $\PP^n$ of degrees $d_1, \dots, d_r$, we have
	\[
		\omega_X = \Oc_X\left(-n-1 + \sum_i d_i\right).
	\]
	We need to use Bertini's theorem in the argument to handle the case where the partial intersections may not be smooth.
\end{ex}

\section{10/16 (Joe Hlavinka) -- The Chow Ring of $\PP^n$ and Applications}

\begin{prop}
	\[
		A(\PP^n) = \ZZ[\zeta] / (\zeta^{n+1})
	\]
	where $[\zeta]$ is the class of a hyperplane.
\end{prop}

\begin{proof}
	Consider the stratification $\pt \subset \PP^1 \subset \PP^2 \subset \dots \subset \PP^n$.
	The open strata are $\AA^0, \AA^1, \dots, \AA^n$.
	Thus each $A^k(\PP^n)$ is generated (as a group) by $[\PP^{n-k}]$.

	To construct the ring structure, note that $A^n(\PP^n) \cong \ZZ$, generated by the class of any point.
	If we intersect a $k$-plane with a general $\ell$-plane, we get a $(n-k-\ell)$-plane.
	In particular, if we multiply a fixed $k$-plane with a general $(n-k)$-plane, we get a point, giving a surjection $A^k(\PP^n) \twoheadrightarrow A^n(\PP^n) \cong \ZZ$.
	Since each $A^k(\PP^n)$ is generated by $[\PP^{n-k}] = [\zeta]^{n-k}$, we see that each $A^k(\PP^n)$ has $[\zeta]^{n-k}$ as a free cyclic generator, yielding the claim.
\end{proof}

\subsection{First applications}

\begin{cor}
	Every map from $\PP^n$ to a quasiprojective variety $X$ with $\dim X < n$ is constant.
\end{cor}

\begin{proof}
	Let $\phi: \PP^n \twoheadrightarrow X$ be the map (WLOG assumed surjective).
	Let $H$ be a general hyperplane section of $X$ in $\PP^m$, and let $x \in X$ be a point disjoint from $H$.
	Then $\dim \phi\inv(H) \geq n-1$ and $\dim \phi\inv(x) > 0$ (since $\dim \PP^n > \dim X$).
	If $\phi$ is nonconstant, then $\phi^*([\{x\} \cap H]) = [\phi\inv(x)] \cdot [\phi\inv(H)] \neq 0$.
	However, $\{x\} \cap H = \emptyset$, contradiction.
\end{proof}

\begin{thm}[B\'ezout]
	If $X_1, \dots, X_k$ are subvarieties of $\PP^n$ such that $\sum_i \codim X_i \leq n$, and the $X_i$ intersect generically transversally, then
	\[
		\deg (X_1 \cap \dots X_n) = \prod_i \deg (X_i).
	\]
\end{thm}

Remember that $\deg|_{A^n(\PP^n)}$ counts the number of points.
The above version of B\'ezout's theorem generalizes this to intersections of arbitrary dimension.
The classical version of B\'ezout's theorem is the case $n = 2$, $\dim X_1 = \dim X_2 = 1$.
If the $X_i$ are Cohen-Macaulay, we can weaken the hypothesis of generic transversality.

\subsection{The Veronese embedding}

Let $V_{n,d}: \PP^n \to \PP^N$ ($N = \binom{n+d}{n} - 1$) be the Veronese embedding, defined by
\[
	V_{n,d}([Z_0, \dots, Z_n]) = [\dots, Z^I, \dots]
\]
where $Z^I$ ranges over all monomials of degree $d$ in $n+1$ variables.
Equivalently, $V_{n,d}$ is the map induced by the complete linear system $|\Oc_{\PP^n}(d)|$.
Thus, if $H$ is a general hyperplane of $\PP^N$, then $\deg V_{n,d}\inv(H) = d$.

\begin{thm}
	The image of $V_{n,d}$ has degree $d^n$.
\end{thm}

\begin{proof}
	It is not hard to show that $V_{n,d}$ is actually an embedding.
	Thus, for $n$ general hyperplanes $H_1, \dots, H_{n}$ in $\PP^n$, the cardinality of the intersection $V_{n,d}(\PP^n) \cap H_1 \cap \dots \cap H_{n}$ is the cardinality of $V_{n,d}\inv(H_1) \cap \dots \cap V_{n,d}\inv(H_n)$.
	Each $V_{n,d}\inv(H_i)$ is a degree $d$ hypersurface in $\PP^n$.
	It follows that the cardinality of the intersection is $d^n$.
\end{proof}

Consider in particular $V_{2,2}: \PP^2 \to \PP^5$.
If $C \subset \PP^2$ is a plane curve of degree $d$, then $V_{2,2}(C)$ has degree $2d$ (in $\PP^5$).
This follows from a similar argument to the above (namely, for a general hyperplane $H \subset \PP^5$, the intersection $V_{2,2}(C) \cap H$ can be identified with $C \cap V_{2,2}\inv(H)$, which has cardinality $2$).
In particular, $V_{2,2}(\PP^2)$ only contains curves of even degree.

\section{10/23 (Joe Hlavinka) -- Continued}

\subsection{More on the Veronese embedding}

Every degree $d$ irreducible curve in $\PP^d$ is isomorphic to the image of the Veronese map $V_{1,d}: \PP^1 \to \PP^d$.
We call this the \emph{rational normal curve} in $\PP^d$.
The proof may be discussed at the end.

\subsection{Degrees of dual hypersurfaces}

Let $X = V(F) \subset \PP^n$ be a smooth degree $d$ hypersurface.
We can define the \emph{Gauss map} $G_X: \PP^n \to (\PP^n)^*$ (where $(\PP^n)^*$ is the dual projective space) via
\[
	p \mapsto \left[ \frac{\partial F}{\partial x_0}(p) : \dots : \frac{\partial F}{\partial x_n}(p)\right].
\]
Euler's theorem on homogeneous functions implies that this is defined on all of $\PP^n$ (since $F$ and the $\partial F / \partial x_i$ cannot all be simultaneously zero).
Note that $\partial F / \partial x_i$ is a degree $(d-1)$ polynomial.
For $p \in X$, $G_X(p)$ classifies the tangent space $T_p X$ (since the image of $G_X(p)$ is the ``gradient of $F$ at $p$,'' and $T_p X$ is the orthogonal to this).

Write $X^* = G_X(X)$.
Note that $X^*$ is typically not smooth.

\begin{ex}
	A general quartic $C \subset \PP^2$ has $28$ bitangents (i.e.\ tangent lines passing through $p_1, p_2 \in X$).
	These correspond to nodes on $C^*$.
\end{ex}

Later in the book, it is shown that $G_X$ is finite and birational.
In particular, by Zariski's main theorem, if $X$ is a curve then either $G_X: X \to X^*$ is an isomorphism or $X^*$ is not smooth.

\begin{prop}
	If $X \subset \PP^n$ is a degree $d$ hypersurface, $X^*$ has degree $d (d-1)^{n-1}$.
\end{prop}

\begin{proof}
	Since $X$ is a hypersurface, $X^*$ must be a hypersurface as well.
	Thus $X^*$ has dimension $n-1$, and we can compute $\deg X^*$ as the cardinality of $X^* \cap H_1 \cap \dots \cap H_{n-1}$ for general hyperplanes $H_i \subset (\PP^n)^*$.
	Because these are general and $X$ is birational, we can replace this by computing the cardinality of $X \cap G_X\inv(H_1) \cap \dots \cap G_X\inv(H_{n-1})$.
	Each $G_X\inv(H_i)$ has degree $d - 1$, so we compute $\deg X^* = d (d-1)^{n-1}$.
\end{proof}

\begin{ex}
	Let $X \subset \PP^2$ be a smooth cubic curve.
	Then $X^*$ has degree $6$ in $(\PP^2)^*$, so through a generic $p \in \PP^2$, there exist $6$ tangent lines to $X$ hitting $p$.
	(The tangent lines are tangent to $X$ at other points.)
\end{ex}

\subsection{Products of projective spaces}

\begin{thm}
	\[
		A(\PP^n \times \PP^m) = \ZZ[x_1, x_2] / (x_1^{n+1}, x_2^{m+1}) = \ZZ[x_1] / (x_1^{n+1}) \otimes \ZZ[x_2] / (x_2^{m+1}).
	\]
\end{thm}

\begin{proof}
	By results of Totaro, an affine stratification gives a basis for the Chow group.
	We have a stratification of $\PP^n \times \PP^m$ by $\AA^i \times \AA^j$ for $0 \leq i \leq n$, $0 \leq j \leq m$.
	One can check that $x_1 = [\PP^n \times \{ 0 \}]$ and $x_2 = [\{ 0 \} \times \PP^m]$ give generators satisfying the stated relations.
\end{proof}

In general there is not a K\"unneth formula for Chow rings.
This is not true (even for elliptic curves).
However, Totaro shows that smooth varieties with affine stratifications satisfy K\"unneth formulas for Chow rings.

Let $\Sigma_{r,s}: \PP^r \times \PP^s \to \PP^{(r+1)(s+1)-1}$ be the Segre embedding, given by
\[
	\Sigma_{r,s}([x_0 : \dots : x_r], [y_0 : \dots : y_s]) \mapsto [\dots : x_i y_j : \dots].
\]
We can also describe $\Sigma_{r,s}$ as the map corresponding to the line bundle $\Oc_{\PP^r}(1) \boxtimes \Oc_{\PP^s}(1)$ on $\PP^r \times \PP^s$.

\begin{prop}
	The degree of the image of $\Sigma_{r,s}$ is $\binom{r+s}{r} = \binom{r+s}{s}$.
\end{prop}

\begin{proof}
	As above, we can compute this degree as the cardinality of $\Sigma_{r,s}\inv(H_1) \cap \dots \cap \Sigma_{r,s}\inv(H_{r+s})$ for general hyperplanes $H_i$.
	Each class $[\Sigma_{r,s}\inv(H_i)]$ is $x_1 + x_2$ (this is easiest to see through the line bundle description of $\Sigma_{r,s}$).
	This is given by $(x_1 + x_2)^{r+s}$, and the only nonzero term in this expansion is
	\[
		\binom{r+s}{r} x_1^r x_2^s = \binom{r+s}{r} [\pt]. \qedhere
	\]
\end{proof}

\begin{prop}
	Let $\Delta: \PP^n \to \PP^n \times \PP^n$ be the diagonal embedding.
	Then
	\[
		[\Delta] = \sum_{i=0}^n x_1^i x_2^{n-i}.
	\]
\end{prop}

\begin{proof}
	Since $[\Delta]$ has dimension $n$, we can write
	\[
		[\Delta] = \sum_{i=0}^n c_i x_1^i x_2^{n-i}.
	\]
	for some $c_i$ which we seek to compute.
	Note that $c_i$ is the degree of $[\Delta] \cdot x_1^{n-i} x_2^i$.
	So we seek to compute the cardinality of the intersection $\Delta \cap (H_1^{n-i} \times H_2^i)$ of $\Delta$ with a product of generic linear subspaces of $\PP^n$.
	Note that 
	\[
		\Delta\inv(H_1^{n-i} \times H_2^i) = H_1^{n-i} \cap H_2^i = \pt,
	\]
	so $c_i = \deg(\pt) = 1$ for all $i$.
\end{proof}

We can use this to compute the degrees of graphs of morphisms from $\PP^n \to \PP^n$.
This can be used to compute the number of fixed points of a morphism.
We may discuss this in more detail at the start of next time.

\section{10/30 (Joe Hlavinka) -- Concluded}

\subsection{Some refreshers}

Recall that
\[
	A^*(\PP^r \times \PP^s) = A^*(\PP^r) \otimes_\ZZ A^*(\PP^s) = \ZZ[\alpha_1, \alpha_2] / (\alpha_1^{r+1}, \alpha_2^{s+1}).
\]
Here each $\alpha_i$ is the pullback of a hyperplane class.
In general, the K\"unneth formula fails, but it is true in this case (or more generally when both factors have an affine stratification).

Write $\sigma_{r,s}: \PP^r \times \PP^s \to \PP^{(r+1)(s+1)-1}$ for the Segre map, and let $\Sigma_{r,s}$ be the image of the Segre map.
We computed
\[
	\deg \Sigma_{r,s} = \binom{r+s}{r}
\]
by noting that $\Sigma_{r,s}$ is induced by the very ample line bundle $\Oc_{\PP^r}(1) \times \Oc_{\PP^s}(1)$, and therefore $\alpha_1 + \alpha_2$ is the pullback of a hyperplane class.

If $\Delta \subset \PP^r \times \PP^r$ is the image of the diagonal $\delta: \PP^r \to \PP^r \times \PP^r$, then
\[
	[\Delta] = \sum_{i=0}^r \alpha_1^{r-i} \alpha_2^i.
\]
We computed this by writing 
\[
	[\Delta] \cdot \alpha_1^i \alpha_2^{r-i} = [\Delta \cap (\Lambda_1 \times \Lambda_2)] = \delta_* [\Lambda_1 \cap \Lambda_2]
\]
for generic linear subspaces $\Lambda_j$, and then noting that each such intersection consists of a single point.
Similar computations hold for diagonals for a larger number of factors.

\subsection{The class of a graph}

Consider a morphism $\phi: \PP^r \to \PP^s$, given by
\[
	X = [X_0 : \dots : X_r] \mapsto [F_0(X) : \dots : F_s(X)]
\]
where each $F_i$ is homogeneous of degree $d$.

\begin{prop}
	The class of the graph of $\phi$ is
	\[
		[\Gamma_\phi] = \sum_{i=0}^r d^i \alpha_1^i \alpha_2^{s-i}.
	\]
\end{prop}

\begin{proof}
	As above, consider $[\Gamma_\phi] \cdot \alpha_1^{r-i} \alpha_2^i$.
	We can identify this with
	\[
		[\Gamma_\phi \cap (\Lambda_1 \cap \Lambda_2)] = [\Lambda_1 \cap \phi\inv(\Lambda_2)]
	\]
	for general hyperplanes $\Lambda_1$ and $\Lambda_2$.
	This intersection should have $d^i$ points.
\end{proof}

After this we tried working on an exercise.

\end{document}
